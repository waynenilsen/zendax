\documentclass[12pt]{article}
\usepackage{amsmath}
\usepackage{amssymb}
\usepackage{graphicx}
\usepackage{hyperref}
\usepackage[latin1]{inputenc}
\usepackage[margin=1in]{geometry}

\newcommand{\half}{\tfrac{1}{2}}

\title{Derivatives with no expiries}
\author{Wayne Nilsen}
\date{2021}

\begin{document}
    \maketitle

    \section{This section may be entirely pointless}

    Originally I wanted this section to show how you can get a payoff today by multiplying by a constant factor that is deeply related to the payoff in the future. That is naturally entirely possible although probably useless in the end.

    The biggest problem that we face with expiries is that it forces LPs and traders alike to fragment liquidity and attention span. Traditionally there is some payoff function $f(x)$ that gives the payoff of a derivative corresponding to some maturity time. That contract is valued at some current time and traded. Turning this problem on its head, rather than proposing a derivative whose terminal value is $f(x)$ we want a derivative with a current value of $f(x)$ and ask how much we would have to pay over time to obtain that exposure.

    Jensen's inequality gives a hint of how to accomplish this. To state it plainly, Jensen's inequality states that when $f$ is convex, $X$ is a random variable and $E$ is the expected value

    $$f(E[X]) \leq E[f(X)]$$

    If we take this one step further and consider $f$ to be what we want the current value function of our derivative and consider $X$ to be the discounted price of the underlying. To simplify expressions for the time being we will assume that dividends and the risk free rate are zero, $r=q=0$. Because the discounted price of the underlying is a martingale under the risk neutral measure, $f(E_0^\star\left[S_t\right]) = S_0$

    \begin{align*}
        \mathcal{J} &\equiv  \frac{f\left( E_0^\star\left[S_t\right])\right)}{E_0^\star\left[ f(S_t)\right])} \\
        & = \frac{f\left( S_0 \right)}{E_0^\star\left[ f(S_t)\right])} \quad \text{(martingale)}
    \end{align*)}

    Note we drop the $E_0^\star$ and just go with $E$ from now for readability and without loss of generality. If there is a need to specify a filtration other than $\mathcal{F}_0$ it will be specified with a subscript and if there is a need to specify a measure other than the risk neutral measure it will be specified with a superscript and most likely noted.

    If the payoff function for a new derivative is $g(x) = f(x)\mathcal{J}(x)$ then the price is

    \begin{align*}
        E[g(S_t)] &= E[f(S_t) \mathcal{J}(S_t)] \\
        &= E[f(S_t) f(S_0) / E[f(S_t)]] \\
        &= f(S_0) E[f(S_t) / E[f(S_t)]] \\
        &= f(S_0) E[f(S_t)] / E[f(S_t)] \\
        &= f(S_0)
    \end{align*}



\end{document}